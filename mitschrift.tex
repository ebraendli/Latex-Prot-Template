\documentclass[10pt,a4paper]{article}
\usepackage{ngerman}
% \usepackage[english]{babel}
% \usepackage{fullpage}

\usepackage[utf8]{inputenc} 
\usepackage{esa}
\usepackage[top=1in, bottom=1in, left=1in, right=1in]{geometry}
\usepackage{url}
\usepackage{cite}
\usepackage{fancyhdr}
\pagestyle{fancy}

\lhead{Erik Brändli}
\chead{Matrikelnummer / 11713731}
\rhead{\today}

\cfoot{Linsensystem}
\rfoot{\thepage}

\renewcommand{\headrulewidth}{0.4pt}
\renewcommand{\footrulewidth}{0.4pt}

\begin{document}
\thispagestyle{empty}
\title{Beispiel Protokoll Linsensysteme}

\author{{\bfseries Erik Brändli}\\
   (Universität Wien, Österreich\\
   a11713731@univie.ac.at)\\
}
\maketitle
%%%%%%%%%%%%%%%%%%%%%%%%%%%%%%%%%%%%%%%%%%%%%%%%%%%%%%%%%%%%%%%%%%%%%%%%%%%%%%%
%%%%%%%%%%%%%%%%%%%%%%%%%%%%%%%%%%%%%%%%%%%%%%%%%%%%%%%%%%%%%%%%%%%%%%%%%%%%%%%
\begin{abstract}
Die Kurzfassung sollte eine kurze Zusammenfassung der Arbeit in ca. 200 bis 
300 Wörtern sein. Und für Praktische Beispiele zu E1 sollten die Dokumentationen nicht 3 Seiten überschreiten.\\ 
Unn here is an example of a typo
\end{abstract}

\begin{keywords}
Linsen, Konvex, Konkav, Bessel-Verfahren, Linsengleichung,..
\end{keywords}

% \tableofcontents % kann optional aktiviert werden

%%%%%%%%%%%%%%%%%%%%%%%%%%%%%%%%%%%%%%%%%%%%%%%%%%%%%%%%%%%%%%%%%%%%%%%%%%%%%%%
\section{Einleitung}
%%%%%%%%%%%%%%%%%%%%%%%%%%%%%%%%%%%%%%%%%%%%%%%%%%%%%%%%%%%%%%%%%%%%%%%%%%%%%%%

In der Einleitung soll die Aufmerksamkeit des Lesers gewonnen und der Inhalt
der Arbeit näher beschrieben werden. Hier sind auch vorwärtsverweise auf spätere
Kapitel erlaubt. 
\newpage
%%%%%%%%%%%%%%%%%%%%%%%%%%%%%%%%%%%%%%%%%%%%%%%%%%%%%%%%%%%%%%%%%%%%%%%%%%%%%%%
\section{Erster Abschnitt}
%%%%%%%%%%%%%%%%%%%%%%%%%%%%%%%%%%%%%%%%%%%%%%%%%%%%%%%%%%%%%%%%%%%%%%%%%%%%%%%

Hier beginnen die eigentlichen Inhalte. Zitate werden in folgender Form 
geschrieben:
"`Es war März oder vielleicht April, und wenn der Schnee in der Petersgatan sich
in Matsch verwandelt haben sollte, habe ich es weder gemerkt, noch hat es mich
sonderlich interessiert. Ich verbrachte meine Zeit hauptsächlich im Bademandel,
über meinen unattraktiven neuen Computer gebeugt, hinter dichtgewebten schwarzen
Vorhängen, die mich gegen die Sonne und vor allem gegen die Außenwelt
abschirmten. Ich kratzte die monatlichen Raten für meinen PC zusammen, der in
drei Jahren abgezahlt werden sollte. Damals wusste ich noch nicht, dass ich die
Raten nur noch ein Jahr lang würde aufbringen müssen. Da nämlich würde ich Linux
geschrieben und sehr viel mehr Leute als Sara und Lars würden es gesehen haben.
Und Peter Anvin, der heute wie ich bei Transmeta arbeitet, würde eine Sammlung
im Internet gestartet haben, um das Geld für meinen Computer
aufzutreiben."'\cite{linus}

Siehe auch \cite{wp:just_for_fun}.
%%%%%%%%%%%%%%%%%%%%%%%%%%%%%%%%%%%%%%%%%%%%%%%%%%%%%%%%%%%%%%%%%%%%%%%%%%%%%%%
\section{Zweiter Abschnitt}
%%%%%%%%%%%%%%%%%%%%%%%%%%%%%%%%%%%%%%%%%%%%%%%%%%%%%%%%%%%%%%%%%%%%%%%%%%%%%%%

%%%%%%%%%%%%%%%%%%%%%%%%%%%%%%%%%%%%%%%%%%%%%%%%%%%%%%%%%%%%%%%%%%%%%%%%%%%%%%%
\section{und so weiter}
%%%%%%%%%%%%%%%%%%%%%%%%%%%%%%%%%%%%%%%%%%%%%%%%%%%%%%%%%%%%%%%%%%%%%%%%%%%%%%%

%%%%%%%%%%%%%%%%%%%%%%%%%%%%%%%%%%%%%%%%%%%%%%%%%%%%%%%%%%%%%%%%%%%%%%%%%%%%%%%
\section{Zusammenfassung und Ausblick}
%%%%%%%%%%%%%%%%%%%%%%%%%%%%%%%%%%%%%%%%%%%%%%%%%%%%%%%%%%%%%%%%%%%%%%%%%%%%%%%

%%%%%%%%%%%%%%%%%%%%%%%%%%%%%%%%%%%%%%%%%%%%%%%%%%%%%%%%%%%%%%%%%%%%%%%%%%%%%%%
%%%%%%%%%%%%%%%%%%%%%%%%%%%%%%%%%%%%%%%%%%%%%%%%%%%%%%%%%%%%%%%%%%%%%%%%%%%%%%%
\begin{thebibliography}{10}
\bibitem[Torvalds, Diamond 2001]{linus} Torvalds L., Diamond D.:
"`Just for Fun -- Wie ein Freak die Computerwelt revolutionierte"';
Carl Hanser, München / Wien (2001)
%\bibitem[Wikipedia: Just For Fun 2006]{wp:just_for_fun}
%\url{http://de.wikipedia.org/wiki/Just_for_Fun} (1.1.2007, Wikipedia)
\end{thebibliography}

\end{document}
